%% 
%% Copyright 2019-2020 Elsevier Ltd
%% 
%% This file is part of the 'CAS Bundle'.
%% --------------------------------------
%% 
%% It may be distributed under the conditions of the LaTeX Project Public
%% License, either version 1.2 of this license or (at your option) any
%% later version.  The latest version of this license is in
%%    http://www.latex-project.org/lppl.txt
%% and version 1.2 or later is part of all distributions of LaTeX
%% version 1999/12/01 or later.
%% 
%% The list of all files belonging to the 'CAS Bundle' is
%% given in the file `manifest.txt'.
%% 
%% Template article for cas-dc documentclass for 
%% double column output.

%\documentclass[a4paper,fleqn,longmktitle]{cas-dc}
\documentclass[a4paper,fleqn]{cas-dc}

%\usepackage[authoryear,longnamesfirst]{natbib}
%\usepackage[authoryear]{natbib}
\usepackage[authoryear]{natbib}

\begin{document}
\let\WriteBookmarks\relax
\def\floatpagepagefraction{1}
\def\textpagefraction{.001}
\shorttitle{PALEO-SEAL}
\shortauthors{J Drechsel et~al.}

\title [mode = title]{PALEO-SEAL: an easily deployable tool for the communication and sharing of Holocene sea-level data.}                      

\author[1]{Jan Drechsel}
\ead{jpmdrechsel@googlemail.com}
\credit{Development of the tool, work on example dataset}
\address[1]{MARUM, Center for Marine Environmental Sciences, University of Bremen, Germany}

\author[2]{Nicole Khan}
\ead{nskhan@hku.hk}
\credit{Work on database template and contribution to paper writing}
\address[2]{Department of Earth Sciences, University of Hong Kong, Hong Kong}

\author[1]{Alessio Rovere}
\cormark[1]
\ead{arovere@marum.de}
\credit{Writing of the paper, supervision on tool development}

\begin{abstract}
Lorem ipsum dolor sit amet, consectetur adipiscing elit. Nunc aliquet nunc risus, vitae porta justo semper lobortis. In tincidunt lacus nec felis dapibus, vitae vehicula elit suscipit. Nullam non lorem sed erat commodo euismod. Pellentesque non elit aliquam, dapibus ante a, commodo elit. Quisque in dignissim elit, rutrum dignissim lorem. Praesent interdum ac nibh et tempus. Donec rutrum rhoncus leo, eget egestas ex pharetra sit amet. Nulla accumsan commodo imperdiet.
\end{abstract}

\begin{keywords}
Sea-level databases \sep Visualization \sep Web interface \sep 
\end{keywords}

\maketitle

\section{Introduction}
The standardization of Holocene sea-level proxies has been a recurrent theme in coastal Quaternary Science research. While it was theorized and approaches have been proposed at least since the early 80s \citep{shennan1982,shennan1983,VanDePlassche1986}, only recent works have established a comprehensive framework for the standardization of sea-level data \citep{khan2019}. The sea-level data standardization efforts were in part elicited by both several IGCP (International Geological Correlation Programme, later renamed as the International Geoscience Programme) projects and the INQUA-PAGES project PALSEA (Palaeo-Constraints on Sea-Level Rise). 

A paper stemming from the PALSEA community \citep{dusterhus2016} highlights that there are key elements to be considered when compiling a sea-level database are Accessibility, Transparency, Trust, Availability, Continuity, Completeness, and Communication of
content. This set of properties is abbreviated into ATTAC\textsuperscript{3}. ``Communication of content'',  according to \citet{dusterhus2016}, means that interfaces for visualization, and standardized protocols for data extraction need to be implemented in order to allow users from different disciplines to easily visualize and export data of interest. 

In this short note, we present one tool designed to meet such criteria, called PALEO-SEAL. The tool makes use of a mySQL version of the sea-level data template of \citet{khan2019}. Installed on any web server supporting PHP and with few simple steps to set it up, it can be used to create a webpage to explore, plot and download Holocene sea-level data. 

\section{PALEO-SEAL description}
The core of PALEO-SEAL are two main data visualization options. One is a map, where points are clustered and de-clustered at different zoom levels. Within the map, data can be filtered either by a drop-down menu or via a select tool directly within the map. The drop-down menu allows to select between: data type (type of sea-level indicator), Region, Subregion, Reference, Publication year, or Dating method. Data can also be filtered via a ``draw rectangle'' tool (Figure \ref{fig:1}). Once a subset of data is selected, it is possible to visualize it in a data explorer interface (Figure \ref{fig:2}). The data explorer interface is composed by an age/elevation graph (with adjustable X and Y axes) and a simplified table that previews the sea-level data plotted. 

The data explorer interface has the same data filtering options as the map, and the two interfaces are linked: what is selected on the map will appear in the data interface and vice-versa. From both map and data explorer, it is possible to create a list of dataponits to be exported. Once filtering is over, an ``Export'' button allows to download the selected data as a *.csv file, compliant with the \citet{khan2019} template.

\begin{figure*}
	\centering
	\includegraphics[width=0.8\textwidth]{figs/Figure1.png}
	\caption{}
	\label{fig:1}
\end{figure*}

\begin{figure*}
	\centering
	\includegraphics[width=0.8\textwidth]{figs/Figure2.png}
	\caption{}
	\label{fig:2}
\end{figure*}

\section{Installing PALEO-SEAL}



\section{Bibliography styles}


\section{Floats}

\begin{figure}
	\centering
		\includegraphics[scale=.75]{figs/Fig1.pdf}
	\caption{The evanescent light - $1S$ quadrupole coupling
	($g_{1,l}$) scaled to the bulk exciton-photon coupling
	($g_{1,2}$). The size parameter $kr_{0}$ is denoted as $x$ and
	the \PMS is placed directly on the cuprous oxide sample ($\delta
	r=0$, See also Table \protect\ref{tbl1}).}
	\label{FIG:1}
\end{figure}



\section[Theorem and ...]{Theorem and theorem like environments}

\section{Bibliography}

%\item Parenthetical: \verb+\citep{WB96}+ produces (Wettig \& Brown, 1996).
%\item Textual: \verb+\citet{ESG96}+ produces Elson et al. (1996).
%\item An affix and part of a reference:\break
%\verb+\citep[e.g.][Ch. 2]{Gea97}+ produces (e.g. Governato et
%&al., 1997, Ch. 2).

\appendix
\section{My Appendix}
Appendix sections are coded under \verb+\appendix+.

\printcredits

%% Loading bibliography style file
%\bibliographystyle{model1-num-names}
\bibliographystyle{cas-model2-names}

% Loading bibliography database
\bibliography{cas-refs}


%\vskip3pt



\end{document}

